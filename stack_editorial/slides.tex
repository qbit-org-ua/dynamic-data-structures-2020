\documentclass[pdf,russian,aspectratio=169]{beamer}

\usepackage[T2A]{fontenc}
\usepackage[utf8]{inputenc}
\usepackage[russian]{babel}
\usepackage{booktabs}
\usepackage{minted}
\usepackage{tikz}

\usetikzlibrary{matrix,chains,shapes}

\selectlanguage{russian}

\mode<presentation>{
    \usetheme{CambridgeUS}
    \useoutertheme{infolines}
}

\title{Разбор первых домашних заданий}
\subtitle{очереди, деки, амортизационный анализ}
\author{Артем Оганджанян}
\institute{Q-Bit}
\date{21 апреля 2020 г.}

\begin{document}

\AtBeginSection[]
{
   \begin{frame}
       \frametitle{Оглавление}
       \tableofcontents[currentsection]
   \end{frame}
}
\AtBeginSubsection[]
{
   \begin{frame}
       \frametitle{Оглавление}
       \tableofcontents[currentsection,currentsubsection]
   \end{frame}
}

\begin{frame}
    \titlepage
\end{frame}

\section{Контест на dots}
\subsection{Очередь на массиве}

\begin{frame}[fragile]
  \frametitle{Алгоритм}
  \begin{center}
\begin{tikzpicture} [nodes in empty cells,
      nodes={minimum width=0.8cm, minimum height=0.8cm},
      row sep=-\pgflinewidth, column sep=-\pgflinewidth]
      border/.style={draw}
    
      \matrix(vector)[matrix of nodes,
          row 1/.style={nodes={draw=none, minimum width=0.3cm, minimum height=0.2cm}},
          row 3/.style={nodes={draw=none, minimum width=0.3cm, minimum height=0.2cm}},
          nodes={draw}]
      {
          \tiny{head} \\
          {\color{lightgray}$a_{0}$} &
          {\color{lightgray}$a_{1}$} &
          {\color{lightgray}$a_{2}$} &
          {\color{lightgray}$a_{3}$} &
          {\color{lightgray}$a_{4}$} &
          {\color{lightgray}$a_{5}$} &
          {\color{lightgray}$a_{6}$} &
          {\color{lightgray}$a_{7}$} \\
          \tiny{tail} \\
      };
\end{tikzpicture}
\pause
\begin{tikzpicture} [nodes in empty cells,
      nodes={minimum width=0.8cm, minimum height=0.8cm},
      row sep=-\pgflinewidth, column sep=-\pgflinewidth]
      border/.style={draw}
    
      \matrix(vector)[matrix of nodes,
          row 1/.style={nodes={draw=none, minimum width=0.3cm, minimum height=0.2cm}},
          row 3/.style={nodes={draw=none, minimum width=0.3cm, minimum height=0.2cm}},
          nodes={draw}]
      {
          \tiny{head} \\
          $a_{0}$ &
          $a_{1}$ &
          $a_{2}$ &
          $a_{3}$ &
          {\color{lightgray}$a_{4}$} &
          {\color{lightgray}$a_{5}$} &
          {\color{lightgray}$a_{6}$} &
          {\color{lightgray}$a_{7}$} \\
          & & & & \tiny{tail} \\
      };
\end{tikzpicture}
\pause
\begin{tikzpicture} [nodes in empty cells,
      nodes={minimum width=0.8cm, minimum height=0.8cm},
      row sep=-\pgflinewidth, column sep=-\pgflinewidth]
      border/.style={draw}
    
      \matrix(vector)[matrix of nodes,
          row 1/.style={nodes={draw=none, minimum width=0.3cm, minimum height=0.2cm}},
          row 3/.style={nodes={draw=none, minimum width=0.3cm, minimum height=0.2cm}},
          nodes={draw}]
      {
          & & \tiny{head} \\
          {\color{lightgray}$a_{0}$} &
          {\color{lightgray}$a_{1}$} &
          $a_{2}$ &
          $a_{3}$ &
          {\color{lightgray}$a_{4}$} &
          {\color{lightgray}$a_{5}$} &
          {\color{lightgray}$a_{6}$} &
          {\color{lightgray}$a_{7}$} \\
          & & & & \tiny{tail} \\
      };
\end{tikzpicture}
\pause
\begin{tikzpicture} [nodes in empty cells,
      nodes={minimum width=0.8cm, minimum height=0.8cm},
      row sep=-\pgflinewidth, column sep=-\pgflinewidth]
      border/.style={draw}
    
      \matrix(vector)[matrix of nodes,
          row 1/.style={nodes={draw=none, minimum width=0.3cm, minimum height=0.2cm}},
          row 3/.style={nodes={draw=none, minimum width=0.3cm, minimum height=0.2cm}},
          nodes={draw}]
      {
          & & & & & & \tiny{head} \\
          $a_{0}$ &
          {\color{lightgray}$a_{1}$} &
          {\color{lightgray}$a_{2}$} &
          {\color{lightgray}$a_{3}$} &
          {\color{lightgray}$a_{4}$} &
          {\color{lightgray}$a_{5}$} &
          $a_{6}$ &
          $a_{7}$ \\
          & \tiny{tail} \\
      };
\end{tikzpicture}
  \end{center}
\end{frame}

\begin{frame}[fragile]
  \frametitle{Поля}
  Стек:
  \begin{minted}{C++}
int capacity = DEFAULT_CAPACITY;
int *stack = new int[capacity];
int size_ = 0;
  \end{minted}
  \pause
  \vspace{1em}
  Очередь:
  \begin{minted}{C++}
int capacity = DEFAULT_CAPACITY;
int *queue = new int[capacity];
int head = 0, tail = 0;
  \end{minted}
\end{frame}

\begin{frame}[fragile]
  \frametitle{\mintinline{C++}{int size()}}
  Стек:
  \begin{minted}{C++}
return size_;
  \end{minted}
  \pause
  \vspace{1em}
  Очередь:
  \begin{minted}{C++}
return (tail - head + capacity) % capacity;
  \end{minted}
\end{frame}

\begin{frame}[fragile,t]
  \frametitle{\mintinline{C++}{void change_capacity(int new_capacity)}}
  \begin{onlyenv}<1>
    Стек:
    \begin{minted}{C++}
int *new_stack = new int[new_capacity];
for (int i = 0; i < size_; ++i) {
  new_stack[i] = stack[i];
}
delete[] stack;
stack = new_stack;
capacity = new_capacity;
    \end{minted}
  \end{onlyenv}
  \begin{onlyenv}<2>
    Очередь:
    \begin{minted}{C++}
int *new_queue = new int[new_capacity];
if (tail >= head) {
  for (int i = head; i < tail; ++i) {
    new_queue[i - head] = queue[i];
  }
} else {
  for (int i = head; i < capacity; ++i) {
    new_queue[i - head] = queue[i];
  }
  for (int i = 0; i < tail; ++i) {
    new_queue[capacity - head + i] = queue[i];
  }
}
    \end{minted}
  \end{onlyenv}
  \begin{onlyenv}<3>
    Очередь:
    \begin{minted}{C++}
tail = size();
head = 0;
delete[] queue;
queue = new_queue;
capacity = new_capacity;
    \end{minted}
  \end{onlyenv}
\end{frame}

\begin{frame}[fragile]
  \frametitle{\mintinline{C++}{void push(int value)}}
  Стек:
  \begin{minted}{C++}
ensure_capacity(size_ + 1);
stack[size_++] = value;
  \end{minted}
  \pause
  \vspace{1em}
  Очередь:
  \begin{minted}{C++}
ensure_capacity(size() + 2);
queue[tail] = value;
tail = (tail + 1) % capacity;
  \end{minted}
\end{frame}

\begin{frame}[fragile]
  \frametitle{\mintinline{C++}{int pop()}}
  Стек:
  \begin{minted}{C++}
int result = stack[--size_];
ensure_capacity(size_);
return result;
  \end{minted}
  \pause
  \vspace{1em}
  Очередь:
  \begin{minted}{C++}
int result = queue[head];
head = (head + 1) % capacity;
ensure_capacity(size());
return result;
  \end{minted}
\end{frame}

\subsection{Дек на массиве}

\begin{frame}[fragile]
  \frametitle{\mintinline{C++}{void push_front(int value)}}
  \begin{minted}{C++}
ensure_capacity(size() + 2);
head = (head - 1 + capacity) % capacity;
deque[head] = value;
  \end{minted}
\end{frame}

\begin{frame}[fragile]
  \frametitle{\mintinline{C++}{int pop_back()}}
  \begin{minted}{C++}
tail = (tail - 1 + capacity) % capacity;
int result = deque[tail];
ensure_capacity(size());
return result;
  \end{minted}
\end{frame}

\subsection{Очередь на списке}

\begin{frame}[fragile]
  \frametitle{Односвязный список}
  \begin{center}
    \begin{tikzpicture}[
        every node/.style={
          rectangle split,
          rectangle split parts=2,
          rectangle split horizontal,
          minimum height=14pt
        },
        node distance=2em,
        start chain,
        every join/.style={->, shorten <=-4.5pt}
      ]

      \node[draw, on chain, join] { $a_0$ \nodepart{two} \phantom{$a_0$} };
      \node[draw, on chain, join] { $a_1$ \nodepart{two} \phantom{$a_1$} };
      \node[draw, on chain, join] { $a_2$ \nodepart{two} \phantom{$a_2$} };
      \node[draw, on chain, join] { $a_3$ \nodepart{two} \phantom{$a_3$} };
      \path[<-, draw, shorten >=10pt]
        (chain-1.one north) |- node [at end] {head} ++(-1,0.7);
      \path[<-, draw, shorten >=18pt]
        (chain-4.one north) |- node [at end] {tail} ++(1.5,0.7);
    \end{tikzpicture} 
  \end{center}
\end{frame}

\begin{frame}[fragile]
  \frametitle{Узлы}
  Стек:
  \begin{minted}{C++}
struct stack_node {
  int element;
  stack_node *prev;
};
  \end{minted}
  \pause
  \vspace{1em}
  Очередь:
  \begin{minted}{C++}
struct queue_node {
  int element;
  queue_node *next;
};
  \end{minted}
\end{frame}

\begin{frame}[fragile]
  \frametitle{Поля}
  Стек:
  \begin{minted}{C++}
stack_node *top = NULL;
int size_ = 0;
  \end{minted}
  \pause
  \vspace{1em}
  Очередь:
  \begin{minted}{C++}
queue_node *head = NULL, *tail = NULL;
int size_ = 0;
  \end{minted}
\end{frame}

\begin{frame}[fragile]
  \frametitle{\mintinline{C++}{void push(int value)}}
  Стек:
  \begin{minted}{C++}
stack_node *new_top = new stack_node{value, top};
top = new_top;
++size_;
  \end{minted}
  \pause
  \vspace{1em}
  Очередь:
  \begin{minted}{C++}
queue_node *new_tail = new queue_node{value, NULL};
if (tail != NULL) {
  tail->next = new_tail;
  tail = new_tail;
} else {
  head = tail = new_tail;
}
++size_;
  \end{minted}
\end{frame}

\subsection{Дек на списке}

\begin{frame}[fragile]
  \frametitle{Двусвязный список}
  \begin{center}
    \begin{tikzpicture}[
        every node/.style={
          rectangle split,
          rectangle split parts=3,
          rectangle split horizontal,
          minimum height=14pt
        },
        node distance=2em,
        start chain,
        every join/.style={<->, shorten <=-4.5pt, shorten >=-4.5pt}
      ]

      \node[draw, on chain, join] { \phantom{$a_0$} \nodepart{two} \phantom{$a_0$} \nodepart{three} \phantom{$a_0$} };
      \node[draw, on chain, join] { \phantom{$a_1$} \nodepart{two} $a_0$ \nodepart{three} \phantom{$a_1$} };
      \node[draw, on chain, join] { \phantom{$a_2$} \nodepart{two} $a_1$ \nodepart{three} \phantom{$a_2$} };
      \node[draw, on chain, join] { \phantom{$a_3$} \nodepart{two} \phantom{$a_3$} \nodepart{three} \phantom{$a_3$} };
      \path[<-, draw, shorten >=4pt]
        (chain-1.two north) |- node [at end] {head} ++(-1,0.7);
      \path[<-, draw, shorten >=24pt]
        (chain-4.two north) |- node [at end] {tail} ++(1.5,0.7);
    \end{tikzpicture} 
  \end{center}
  \pause
  \vspace{1em}
  \begin{center}
    \begin{tikzpicture}[
        every node/.style={
          rectangle split,
          rectangle split parts=3,
          rectangle split horizontal,
          minimum height=14pt
        },
        node distance=2em,
        start chain,
        every join/.style={<->, shorten <=-4.5pt, shorten >=-4.5pt}
      ]

      \node[draw, on chain, join] { \phantom{$a_0$} \nodepart{two} \phantom{$a_0$} \nodepart{three} \phantom{$a_0$} };
      \node[draw, on chain, join] { \phantom{$a_3$} \nodepart{two} \phantom{$a_3$} \nodepart{three} \phantom{$a_3$} };
      \path[<-, draw, shorten >=4pt]
        (chain-1.two north) |- node [at end] {head} ++(-1,0.7);
      \path[<-, draw, shorten >=24pt]
        (chain-2.two north) |- node [at end] {tail} ++(1.5,0.7);
    \end{tikzpicture} 
  \end{center}
\end{frame}

\begin{frame}[fragile]
  \frametitle{Узлы}
  \begin{minted}{C++}
struct deque_node {
  int element;
  deque_node *next;
  deque_node *prev;
};
  \end{minted}
\end{frame}

\begin{frame}[fragile]
  \frametitle{Поля}
  \begin{minted}{C++}
deque_node *head = NULL, *tail = NULL;
int size_ = 0;

deque() {
  head = new deque_node{0, NULL, NULL};
  tail = new deque_node{0, NULL, NULL};
  head->next = tail;
  tail->prev = head;
}
  \end{minted}
\end{frame}

\begin{frame}[fragile]
  \frametitle{\mintinline{C++}{void push(deque_node *next, int value)}}
  \begin{minted}{C++}
deque_node *node = new deque_node{value, next, next->prev};
next->prev = node;
node->prev->next = node;
++size_;
  \end{minted}
\end{frame}

\begin{frame}[fragile]
  \frametitle{\mintinline{C++}{int pop(deque_node *node)}}
  \begin{minted}{C++}
auto [result, next, prev] = *node;
delete node;
next->prev = prev;
prev->next = next;
--size_;
return result;
  \end{minted}
\end{frame}

\begin{frame}[fragile]
  \frametitle{\mintinline{C++}{push}}
  \begin{minted}{C++}
void push_front(int value) {
  push(head->next, value);
}

void push_back(int value) {
  push(tail, value);
}
  \end{minted}
\end{frame}

\begin{frame}[fragile]
  \frametitle{\mintinline{C++}{pop}}
  \begin{minted}{C++}
int pop_front() {
  return pop(head->next);
}

int pop_back() {
  return pop(tail->prev);
}
  \end{minted}
\end{frame}

\begin{frame}[fragile]
  \frametitle{\mintinline{C++}{front, back}}
  \begin{minted}{C++}
int front() {
  return head->next->element;
}

int back() {
  return tail->prev->element;
}
  \end{minted}
\end{frame}

\begin{frame}[fragile]
  \frametitle{Бонус}
  Список:
  \begin{minted}{C++}
deque_node *head = NULL, *tail = NULL;
int size_ = 0;
  \end{minted}
  \pause
  \vspace{1em}
  Циклический список:
  \begin{minted}{C++}
deque_node *fake = NULL;
int size_ = 0;
deque() {
  fake = new deque_node{0, NULL, NULL};
  fake->next = fake;
  fake->prev = fake;
}
  \end{minted}
\end{frame}

\section{Асимптотика вектора}
\subsection{Увеличение массива}

\begin{frame}[fragile,t]
  \frametitle{Интуиция}

  $\operatorname{capacity}=2$

  \begin{onlyenv}<2-3>
    \vspace{0.5em}
    \begin{center}
      \begin{tikzpicture} [
          nodes in empty cells,
          nodes={minimum width=0.8cm, minimum height=0.8cm},
          row sep=-\pgflinewidth, column sep=-\pgflinewidth
        ]
        \matrix(vector)[
            matrix of nodes,
            row 1/.style={nodes={draw=none, minimum width=0.3cm, minimum height=0.2cm}},
            nodes={draw}
          ]
          {
            \tiny{1} & \tiny{1} & \tiny{2+1} \\
            $1$ & $2$ & $3$ &
            \phantom{$4$} & \phantom{$5$} & \phantom{$6$} & \phantom{$7$} &
            \phantom{$8$} & \phantom{$9$} & \phantom{$10$} & \phantom{$11$} &
            \phantom{$12$} & \phantom{$13$} & \phantom{$14$} & \phantom{$15$} &
            \phantom{$16$} & \phantom{$17$} \\
          };
      \end{tikzpicture}
    \end{center}
    $n=3, \operatorname{time}=3+2=5$
  \end{onlyenv}

  \begin{onlyenv}<3>
    \vspace{0.5em}
    \begin{center}
      \begin{tikzpicture} [
          nodes in empty cells,
          nodes={minimum width=0.8cm, minimum height=0.8cm},
          row sep=-\pgflinewidth, column sep=-\pgflinewidth
        ]
        \matrix(vector)[
            matrix of nodes,
            row 1/.style={nodes={draw=none, minimum width=0.3cm, minimum height=0.2cm}},
            nodes={draw}
          ]
          {
            \tiny{1} &
            \tiny{1} &
            \tiny{2+1} &
            \tiny{1} &
            \tiny{4+1} \\
            $1$ & $2$ & $3$ & $4$ & $5$ & \phantom{$6$} &
            \phantom{$7$} & \phantom{$8$} & \phantom{$9$} & \phantom{$10$} &
            \phantom{$11$} & \phantom{$12$} & \phantom{$13$} & \phantom{$14$} &
            \phantom{$15$} & \phantom{$16$} & \phantom{$17$} \\
          };
      \end{tikzpicture}
    \end{center}
    $n=5, \operatorname{time}=5+4+2=11$
  \end{onlyenv}

  \begin{onlyenv}<4-5>
    \vspace{0.5em}
    \begin{center}
      \begin{tikzpicture} [
          nodes in empty cells,
          nodes={minimum width=0.8cm, minimum height=0.8cm},
          row sep=-\pgflinewidth, column sep=-\pgflinewidth
        ]
        \matrix(vector)[
            matrix of nodes,
            row 1/.style={nodes={draw=none, minimum width=0.3cm, minimum height=0.2cm}},
            nodes={draw}
          ]
          {
            \tiny{1} &
            \tiny{1} &
            \tiny{2+1} &
            \tiny{1} &
            \tiny{4+1} &
            \tiny{1} &
            \tiny{1} &
            \tiny{1} &
            \tiny{8+1} \\
            $1$ & $2$ & $3$ & $4$ & $5$ & $6$ & $7$ & $8$ &
            $9$ & \phantom{$10$} &
            \phantom{$11$} & \phantom{$12$} & \phantom{$13$} & \phantom{$14$} &
            \phantom{$15$} & \phantom{$16$} & \phantom{$17$} \\
          };
      \end{tikzpicture}
    \end{center}
    $n=9, \operatorname{time}=9+8+4+2=23$
  \end{onlyenv}

  \begin{onlyenv}<5>
    \vspace{0.5em}
    \begin{center}
      \begin{tikzpicture} [
          nodes in empty cells,
          nodes={minimum width=0.8cm, minimum height=0.8cm},
          row sep=-\pgflinewidth, column sep=-\pgflinewidth
        ]
        \matrix(vector)[
            matrix of nodes,
            row 1/.style={nodes={draw=none, minimum width=0.3cm, minimum height=0.2cm}},
            nodes={draw}
          ]
          {
            \tiny{1} &
            \tiny{1} &
            \tiny{2+1} &
            \tiny{1} &
            \tiny{4+1} &
            \tiny{1} &
            \tiny{1} &
            \tiny{1} &
            \tiny{8+1} &
            \tiny{1} &
            \tiny{1} &
            \tiny{1} &
            \tiny{1} &
            \tiny{1} &
            \tiny{1} &
            \tiny{1} &
            \tiny{16+1} \\
            $1$ & $2$ & $3$ & $4$ & $5$ & $6$ & $7$ & $8$ &
            $9$ & $10$ & $11$ & $12$ & $13$ & $14$ & $15$ & $16$ & $17$ \\
          };
      \end{tikzpicture}
    \end{center}
    $n=17, \operatorname{time}=17+16+8+4+2=47$
  \end{onlyenv}
\end{frame}

\begin{frame}
  \frametitle{Формула}
  \[
    n = 2^k + 1
  \]
  \pause
  \[
    S_n=\sum_{i=1}^n b_i = \frac{b_1(q^n-1)}{q-1}
  \]
  \pause
  \begin{align*}
    \operatorname{time}(n) &= n + \sum_{i=1}^k 2^i\\
    &= n + 2(2^k-1)\\
    &=n+2(n-2)\\
    &=3n-4\\
    &=O(n)
  \end{align*}
\end{frame}

\subsection{Уменьшение массива}

\begin{frame}[fragile]
  \frametitle{Доказательство}
  Рассмотрим операции между двумя изменениями размера массива.

  \pause
  $n$ элементов с реальным размером массива $2n$.
  \begin{center}
    \begin{tikzpicture} [
        nodes in empty cells,
        nodes={minimum width=0.8cm, minimum height=0.8cm},
        row sep=-\pgflinewidth, column sep=-\pgflinewidth
      ]
      \matrix(vector)[
          nodes in empty cells,
          matrix of nodes,
          nodes={draw,anchor=center}
        ]
        {
          &
          &
          &
          $\frac{n}{2}$ &
          &
          &
          &
          $n$ &
          &
          &
          &
          &
          &
          &
          &
          $2n$ \\
        };
    \end{tikzpicture}
  \end{center}

  \pause
  Рассмотрим только операции удаления или только операции добавления.

  \begin{itemize}
    \pause
    \item Стало $\frac{n}{2}$. \pause
      Потратили времени $\frac{n}{2} + \frac{n}{2} = n$
    \pause
    \item Стало $2n$. \pause Потратили времени $n+2n=3n$.
  \end{itemize}
\end{frame}

\begin{frame}
  \frametitle{Конец!}
  Вопросы?
\end{frame}

\end{document}
