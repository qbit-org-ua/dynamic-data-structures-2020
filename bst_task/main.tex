\documentclass[12pt]{article}

\usepackage{amsmath}
\usepackage{amssymb}
\usepackage{amsthm}
\usepackage{mathtext}
\usepackage{mathtools}
\usepackage[T1,T2A]{fontenc}
\usepackage[utf8]{inputenc}
\usepackage[left=2cm,right=2cm,top=2cm,bottom=2cm]{geometry}
\usepackage{microtype}
\usepackage{enumitem}
\usepackage{bm}
\usepackage{cancel}
\usepackage{proof}
\usepackage{epigraph}
\usepackage{titlesec}
\usepackage[dvipsnames]{xcolor}
\usepackage{stmaryrd}
\usepackage{cellspace}
\usepackage{cmll}
\usepackage{multirow}
\usepackage{booktabs}
\usepackage{tikz}
\usepackage{caption}
\usepackage{wrapfig}
\usepackage{minted}
\usepackage{svg}

\usepackage[hidelinks]{hyperref}

\usepackage[russian]{babel}
\selectlanguage{russian}

\hypersetup{%
    colorlinks=true,
    linkcolor=blue
}

\DeclareMathOperator{\StudentId}{StudentId}
\DeclareMathOperator{\StudentName}{StudentName}
\DeclareMathOperator{\GroupId}{GroupId}
\DeclareMathOperator{\GroupName}{GroupName}
\DeclareMathOperator{\CourseId}{CourseId}
\DeclareMathOperator{\CourseName}{CourseName}
\DeclareMathOperator{\LecturerId}{LecturerId}
\DeclareMathOperator{\LecturerName}{LecturerName}
\DeclareMathOperator{\Mark}{Mark}

\pagenumbering{gobble}

\title{Теоретическое домашнее задание на бинарные деревья поиска}
\author{}
\date{\vspace{-5ex}}

\begin{document}

\usetikzlibrary{arrows.meta,positioning,matrix}

\maketitle

Для каждой вершины бинарного дерева поиска верно следующее: ключ, хранящийся в
вершине, меньше ключей всех вершин левого поддерева вершины, и больше ключей
всех вершин правого поддерева вершины.

Рассмотрим такое бинарное дерево поиска, которое кроме ключей хранит в вершинах
\emph{вес}. Для веса будет поддерживаться следующий инвариант: вес любой вершины
меньше веса вершин левого и правого поддеревьев. То есть, "<тяжёлые"> вершины
будут ниже "<лёгких">. На рисунке \ref{example} приведён пример такого дерева,
чёрным цветом обозначены ключи, синим "--- веса.

\begin{figure}[h]
  \centering
  \begin{tikzpicture}[->,>=Stealth,every path/.style={draw=black,thick}]
    \centering
    \begin{scope}[every node/.style={circle,thick,draw},align=center]
      \node (A7) at (0,0) {7, \color{blue}{1}};
      \node (A3) at (-2,-1.5) {3, \color{blue}{2}};
      \node (A9) at (2,-1.5) {9, \color{blue}{5}};
      \node (A2) at (-3,-3) {2, \color{blue}{3}};
      \node (A5) at (-1,-3) {5, \color{blue}{4}};
      \node (A8) at (1,-3) {8, \color{blue}{9}};
      \node (A1) at (-4,-4.5) {1, \color{blue}{6}};
      \node (A4) at (-2,-4.5) {4, \color{blue}{8}};
      \node (A6) at (0,-4.5) {6, \color{blue}{7}};
    \end{scope}

    \path [->]
      (A7) edge (A3) edge (A9)
      (A3) edge (A2) edge (A5)
      (A9) edge (A8)
      (A2) edge (A1)
      (A5) edge (A4) edge (A6);
  \end{tikzpicture}
  \caption{Пример дерева с весами.}
  \label{example}
\end{figure}

\section{\texorpdfstring{Единственность}{Task 1}}
Пускай дано $n$ вершин $(x_i, y_i)$, $x_i$ "--- ключ вершины $i$, $y_i$ "--- вес
вершины $i$, все ключи различны, все веса различны.
Докажите, что существует единственное дерево, состоящее из этих вершин,
удовлетворяющее описанным свойствам.

\section{\texorpdfstring{Разрез}{Task 2}}
Разрез "--- операция, которая получает на вход дерево и $x$, и возвращает два
дерева. В первом дереве все ключи должны быть меньше $x$, во втором все ключи
должны быть не меньше $x$.  Пример разреза дерева с рисунка \ref{example} по
ключу $5$ приведён на рисунке \ref{split}.

Предложите алгоритм разреза дерева по ключу. Какова сложность вашего алгоритма?

\begin{figure}[h]
  \centering
  \begin{tikzpicture}[->,>=Stealth,every path/.style={draw=black,thick}]
    \centering
    \begin{scope}[every node/.style={circle,thick,draw},align=center]
      \node (A3) at (-4,0) {3, \color{blue}{2}};
      \node (A2) at (-6,-1.5) {2, \color{blue}{3}};
      \node (A1) at (-5,-3) {1, \color{blue}{6}};
      \node (A6) at (-2,-1.5) {6, \color{blue}{7}};
      \node (A4) at (-3,-3) {4, \color{blue}{8}};

      \node (A7) at (3,0) {7, \color{blue}{1}};
      \node (A5) at (1,-1.5) {5, \color{blue}{4}};
      \node (A9) at (5,-1.5) {9, \color{blue}{5}};
      \node (A8) at (4,-3) {8, \color{blue}{9}};
    \end{scope}

    \path [->]
      (A7) edge (A5) edge (A9)
      (A3) edge (A2) edge (A6)
      (A9) edge (A8)
      (A2) edge (A1)
      (A6) edge (A4);
  \end{tikzpicture}
  \caption{Пример разреза дерева с рисунка \ref{example} по ключу $5$.}
  \label{split}
\end{figure}

\section{\texorpdfstring{Слияние}{Task 3}}
Слияние "--- операция, которая на вход получает два дерева, причём все ключи
первого дерева меньше ключей второго дерева, и возвращает дерево, которое
содержит в себе все вершины первого и второго деревьев. Например, слияние
деревьев с рисунка \ref{split} даёт в результате дерево с рисунка \ref{example}.
Деревья, для которых можно эффективно реализовать такую операцию, называют
\emph{сливаемыми}.

Предложите алгоритм слияния двух деревьев. Какова сложность вашего алгоритма?

\section{\texorpdfstring{Вставка}{Task 4}}
Предложите алгоритм вставки вершины с ключом $x$ и весом $y$ в дерево.
Подсказка: воспользуйтесь предыдущими заданиями.
Какова сложность вашего алгоритма?

\section{\texorpdfstring{Удаление}{Task 5}}
Предложите алгоритм удаления вершины с ключом $x$ из дерева.
Какова сложность вашего алгоритма?

\section{\texorpdfstring{Поиск}{Task 6}}
Предложите алгоритм поиска вершины с ключом $x$ в дереве.
Какова сложность вашего алгоритма?

\section{\texorpdfstring{Реализация*}{Task 7}}
Реализуйте такое дерево на вашем любимом языке программирования. В качестве
весов берите случайные числа. Измерьте время работы операций для деревьев
различных размеров. Опишите результаты измерений.

\end{document}
